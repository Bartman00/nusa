\chapter{Introducción al método de los elementos finitos}

En el método de elementos finitos se considera un cuerpo continuo o sólido, como un ensamble de pequeñas 
subdivisiones llamadas elementos finitos. Estos elementos están interconectados a través de nodos comunes. 
Debido a que la variación real de las variables de campo (desplazamientos, esfuerzos, temperaturas, etc.) 
se desconoce en el continuo, se asume que la variación de estas en el modelo de elemento finito puede ser 
aproximada por una simple función. Estas funciones de aproximación, también llamadas modelos de interpolación, 
son definidas en términos de los valores nodales de las variables de campo.\\

En general, el método de los elementos finitos, consiste en formular un sistema de ecuaciones 
(ecuaciones de equilibrio) para el sistema continuo que ha sido discretizado, donde las incógnitas 
suelen ser los valores nodales de las variables de campo. Luego, se resuelve este sistema de ecuaciones, 
con las consideraciones correspondientes a las condiciones de frontera o valores iniciales que simplifiquen 
el modelo original. La siguiente ecuación muestra, en notación matricial, el sistema de ecuaciones resuelto 
en una formulación de elemento finito.

\begin{equation} \label{eq:fem_simple}
K\,\vec{u} = \vec{P}
\end{equation}

Donde $K$ es la matriz global de rigidez, $\vec{u}$ es el vector de desplazamientos nodales y $\vec{P}$ el 
vector de fuerzas nodales en el sistema.\\

Para problemas lineales, el vector $\vec{u}$ puede ser resuelto de manera sencilla, mediante 
procedimientos básicos del álgebra lineal. Sin embargo, para problemas no lineales, la solución 
tiene que ser obtenida mediante una secuencia de pasos, en el cual cada uno de estos implica la 
modificación de la matriz de rigidez $K$ y/o el vector global de carga $\vec{P}$.\\

En problemas de análisis dinámico, desplazamientos, velocidades, deformaciones, esfuerzos y cargas 
son dependientes del tiempo. Por ello deben incluirse algunos otros términos, por lo que la ecuación 
a resolver viene dada por la expresión siguiente:

\begin{equation}
M\ddot{\vec{u}} + C\dot{\vec{u}} ̇+ K\vec{u} = \bf{P}
\end{equation}

Donde $C\,\dot{u}$ representa las fuerzas viscosas, mismas que deben incluirse cuando el 
sistema esté amortiguado artificialmente y $M$ la matriz de masas.\\